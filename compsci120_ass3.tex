\documentclass[12pt]{article}
\title{
    \textbf{COMPSCI130 — Assignment 3} \\
    \large{University of Auckland}
}
\author{Aidan Webster \qquad aweb904@aucklanduni.ac.nz}
\date{\today}

\usepackage{amssymb}
\usepackage{amsthm}
\usepackage{amsmath}
\usepackage{mdframed}
\usepackage{tocloft}
\usepackage{ifthen}
\usepackage{xparse}
\usepackage{calc}
\usepackage[nomessages]{fp}

% Short commands for common number systems
\newcommand{\N}{\mathbb{R}}
\newcommand{\Z}{\mathbb{R}}
\newcommand{\Q}{\mathbb{Q}}
\newcommand{\R}{\mathbb{R}}
\newcommand{\C}{\mathbb{R}}

% Shorthand commands for all braces
\newcommand\set[1]{\left\lbrace #1 \right\rbrace} % Curly braces
\newcommand\abs[1]{\left| #1 \right|} % Absolute value
\newcommand\brac[1]{\left[ #1 \right]} % Square brackets
\newcommand\parens[1]{\left( #1 \right)} % Normal brackets

% Command for solution box
\DeclareDocumentCommand\sol{ m g } {
    % #1 - solution
    % #2 - leftmargin
     
    \IfNoValueTF{#2} {
        \begin{mdframed}
            \emph{Solution. \\[1pt]} #1
        \end{mdframed}
    } {
        \begin{mdframed}[leftmargin=#2]
            \emph{\underline{Solution}. \\[1pt]} #1
        \end{mdframed}
    }
}

% Command for proof box
\DeclareDocumentCommand\prf{ m g } {
    % #1 - solution
    % #2 - leftmargin

    \IfNoValueTF{#2} {
        \begin{mdframed}
            \begin{proof}[\underline{Proof}]
                $ $\\[3pt] #1
            \end{proof}
        \end{mdframed}
    } {
        \begin{mdframed}[leftmargin=#2]
            \begin{proof}[\underline{Proof}]
                $ $\\[3pt] #1
            \end{proof}
        \end{mdframed}
    }
}

% Command for centered outlined box
\DeclareDocumentCommand\widebox{ m } {
    % #1 - Content

    \makebox[\textwidth + 26pt][r] {
        \fbox{
            % Default textwidth
            \parbox{426pt} {
                #1 
            }
        }
    }
}

% Create sections that can be inline
\makeatletter
\newcommand\inlinesection{\@ifstar\mystarsec\mynostarsec}
\makeatother

\newcommand\mystarsec[1]{
\ifhmode\unskip\fi\textbf{#1}}
\newcommand\mynostarsec[1]{
\ifhmode\unskip\fi\refsetepcounter{section}\textbf{\thesection. #1}}

% Command for question
\DeclareDocumentCommand\question{ m m g g } {
    % #1 - Question num/letter
    % #2 - Question
    % #3 - Solution
    % #4 - Proof

    \noindent
    \begin{mdframed}[hidealllines=true, leftmargin=0pt, innertopmargin=0pt, innerbottommargin=0pt, skipabove=0pt, skipbelow=0pt]        
        \makebox[-5pt][r]{\Large{#1)}}
        \inlinesection*{\textnormal{\large{#2}}}
    \end{mdframed}

    \addcontentsline{toc}{section}{Question #1}

    \IfNoValueF{#3} {
        \IfBlankF{#3} {
            \sol{#3}{10}
        }
    }
    \IfNoValueF{#4} {
        \IfBlankF{#4} {
            \sol{#4}{10}
        }
    }
}

% Command for subquestions (e.g. a) b) c) etc)
\DeclareDocumentCommand\subquestion{ m m g g } {
    % #1 - Question num/letter
    % #2 - Question
    % #3 - Solution
    % #4 - Proof

    \noindent
    \begin{mdframed}[hidealllines=true, leftmargin=15pt, innertopmargin=0pt, innerbottommargin=0pt, skipabove=0pt, skipbelow=0pt]        
        \makebox[-5pt][r]{\large{#1)}}
        \inlinesection*{\textnormal{\normalsize{#2}}}
    \end{mdframed}

    \addcontentsline{toc}{subsection}{#1)}

    \IfNoValueF{#3} {
        \IfBlankF{#3} {
            \sol{#3}{25}
        }
    }
    \IfNoValueF{#4} {
        \IfBlankF{#4} {
            \sol{#4}{25}
        }
    }
}

% Command for subsubquestions (e.g. i. ii. iii. etc)
\DeclareDocumentCommand\subsubquestion{ m m g g } {    
    % #1 - Question num/letter
    % #2 - Question
    % #3 - Solution
    % #4 - Proof

    \noindent
    \begin{mdframed}[hidealllines=true, leftmargin=30pt, innertopmargin=0pt, innerbottommargin=0pt, skipabove=0pt, skipbelow=0pt]  
        \makebox[-5pt][r]{\normalsize{#1)}}
        \inlinesection*{\textnormal{\small{#2}}}
    \end{mdframed}

    \addcontentsline{toc}{subsubsection}{#1.}

    \IfNoValueF{#3} {
        \IfBlankF{#3} {
            \sol{#3}{40}
        }
    }
    \IfNoValueF{#4} {
        \IfBlankF{#4} {
            \sol{#4}{40}
        }
    }
}

% Command to make sections have connecting dots in contents
\renewcommand{\cftsecleader}{\cftdotfill{\cftdotsep}}

\begin{document}
\maketitle

\newpage

\tableofcontents

\newpage

% Questions and Answers
\question{1}{Functions}
\subquestion{a}{Let $X = \set{1,2,3,4}$ and $Y = \set{a,b,c,d}$. State with reason which of the rules defined below is (or is not) a function with domain $X$ and codomain $Y$.}
\subsubquestion{i}{$f(4) = a, \ f(2) = d, \ f(1) = c, \ f(3) = b$}
\subsubquestion{ii}{$g(3) = a, \ g(4) = d, \ g(1) = \ c$}
\subsubquestion{iii}{$h(2) = 3, \ h(3) = 1, \ h(1) = 4, \ h(4) = 2$}
\subsubquestion{iv}{$i(4) = c, \ i(1) = b, \ i(2) = a, \ i(3) = d, \ i(4) = a$}

\question{2}{Limits}
\subquestion{a}{Find the following limits. Explain your answer.}
\subsubquestion{i}{$\lim\limits_{n \rightarrow \infty}{\frac{2n}{n \log_{2}{\parens{\frac{1}{n}}}}}$}
\subsubquestion{ii}{$\lim\limits_{n \rightarrow \infty}{\frac{\sqrt{n^5 + 4n + 5} - \sqrt{5n^6 + 3n}}{4n^3 + 81n + 64}}$}
\subsubquestion{iii}{$\lim\limits_{n \rightarrow \infty}{\frac{3^n + n^9}{7^n + n^2 + 5}}$}
\subsubquestion{iv}{$\lim\limits_{n \rightarrow \infty}{\frac{4n + 3}{25 - 5^{\frac{4n + 3}{n^2 + 5n} + 2}}}$}

\question{3}{Algorithms}
\subquestion{a}{Consider the following algorithm: \\
    \widebox{        
        \textbf{Input: } A positive natural number $n$.
        \begin{enumerate}
            \item [1.] If $n \ \% \ 3 = 0$, output $n$ and stop. Otherwise, go to Step 2.
            \item [2.] If $n$ is even, replace $n$ with $n + 1$ and go back to $1.$ Otherwise, go to Step 3.
            \item [3.] Replace $n$ with $n + 2$ and go to Step 1.
        \end{enumerate}  
    } \\
    Will this algorithm run forever? Explain why or why not.
}
\subquestion{b}{Consider two algorithms, called \textbf{Algorithm A} and \textbf{Algorithm B} with the following runtimes:
\begin{gather*}
    \textbf{AlgorithmASteps}(n) = n^{10} + 4n + \log_{2}(n) \\
    \textbf{AlgorithmBSteps}(n) = 5 \log_{2}(n) + 2^{n}
\end{gather*}
}
\subsubquestion{i}{Find the run times for each algorithm for 
$n = 2$ and $n = 8$.}
\subsubquestion{ii}{Which algorithm is more efficient when 
$n < 10$?}
\subsubquestion{iii}{Which algorithm is more efficient for very large values of 
$n$? Explain your answer.}

\newpage

\question{4}{Graphs}
\noindent In lectures, we looked at complete bipartite graphs. Here, we will look at bipartite graphs that are not necessarily complete. A bipartite graph is a simple graph whose vertex set
$V$ can be split into two disjoint nonempty sets
$V_{1}$ and $V_{2}$ for which $V_{1} \cup V_{2} = V$
so that every edge in the graph connects a vertex in
$V_{1}$ with a vertex in $V_{2}$.
\subquestion{a}{Draw two different bipartite graphs, each containing 7 vertices in one set and 2 vertices in another set.}
\subquestion{b}{Consider the following graph $G$:
Is $G$ bipartite? Justify your answer either by drawing $G$ as a bipartite graph with two distinct sets of vertices, or by explaining why $G$ cannot be bipartite.}
\subquestion{c}{San draws a graph and asks Mika if it is bipartite. Mika says the graph is not bipartite because it contains the subgraph, 
$C_3$. Is Mika correct? Can a graph with $C_3$ as a subgraph be bipartite? Explain your answer.}
\subquestion{d}{San draws another graph. This time he notices that the graph contains the subgraph 
$C_4$ and assumes that the graph he has drawn cannot be bipartite. Is he correct? Explain your answers.}
\subquestion{e}{In general, come up with a rule for determining whether a graph with a subgraph 
$C_n$ is bipartite. Explain your answer.}

\end{document}