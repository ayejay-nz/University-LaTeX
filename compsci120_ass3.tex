\documentclass[12pt]{article}
\title{
    \textbf{PAPER — Assignment n} \\
    \large{University of Auckland}
}
\author{Aidan Webster \qquad aweb904@aucklanduni.ac.nz}
\date{\today}

\usepackage{amssymb}
\usepackage{amsthm}
\usepackage{amsmath}
\usepackage{mdframed}
\usepackage{tocloft}
\usepackage{ifthen}
\usepackage{xparse}

% Short commands for common number systems
\newcommand{\N}{\mathbb{R}}
\newcommand{\Z}{\mathbb{R}}
\newcommand{\Q}{\mathbb{Q}}
\newcommand{\R}{\mathbb{R}}
\newcommand{\C}{\mathbb{R}}

% Shorthand commands for all braces
\newcommand\set[1]{\left\lbrace #1 \right\rbrace} % Curly braces
\newcommand\abs[1]{\left| #1 \right|} % Absolute value
\newcommand\brac[1]{\left[ #1 \right]} % Square brackets
\newcommand\parens[1]{\left( #1 \right)} % Normal brackets

% Command for solution box
\DeclareDocumentCommand\sol{ m g } {
    % #1 - solution
    % #2 - leftmargin
     
    \IfNoValueTF{#2} {
        \begin{mdframed}
            \emph{Solution. \\[1pt]} #1
        \end{mdframed}
    } {
        \begin{mdframed}[leftmargin=#2]
            \emph{\underline{Solution}. \\[1pt]} #1
        \end{mdframed}
    }
}

% Command for proof box
\DeclareDocumentCommand\prf{ m g } {
    % #1 - solution
    % #2 - leftmargin

    \IfNoValueTF{#2} {
        \begin{mdframed}
            \begin{proof}[\underline{Proof}]
                $ $\\[3pt] #1
            \end{proof}
        \end{mdframed}
    } {
        \begin{mdframed}[leftmargin=#2]
            \begin{proof}[\underline{Proof}]
                $ $\\[3pt] #1
            \end{proof}
        \end{mdframed}
    }
}

% Command for centered outlined box
\DeclareDocumentCommand\widebox{ m } {
    % #1 - Content

    \makebox[\textwidth + 26pt][r] {
        \fbox{
            % Default textwidth
            \parbox{426pt} {
                #1 
            }
        }
    }
}

% Create sections that can be inline
\makeatletter
\newcommand\inlinesection{\@ifstar\mystarsec\mynostarsec}
\makeatother

\newcommand\mystarsec[1]{
\ifhmode\unskip\fi\textbf{#1}}
\newcommand\mynostarsec[1]{
\ifhmode\unskip\fi\refsetepcounter{section}\textbf{\thesection. #1}}

% Command for question
\DeclareDocumentCommand\question{ m m g g } {
    % #1 - Question num/letter
    % #2 - Question
    % #3 - Solution
    % #4 - Proof

    \noindent
    \begin{mdframed}[hidealllines=true, leftmargin=0pt, innertopmargin=0pt, innerbottommargin=0pt, skipabove=0pt, skipbelow=0pt]        
        \makebox[-5pt][r]{\Large{#1)}}
        \inlinesection*{\textnormal{\emph{\large{#2}}}}
    \end{mdframed}

    \addcontentsline{toc}{section}{Question #1}

    \IfNoValueF{#3} {
        \IfBlankF{#3} {
            \sol{#3}{10}
        }
    }
    \IfNoValueF{#4} {
        \IfBlankF{#4} {
            \sol{#4}{10}
        }
    }
}

% Command for subquestions (e.g. a) b) c) etc)
\DeclareDocumentCommand\subquestion{ m m g g } {
    % #1 - Question num/letter
    % #2 - Question
    % #3 - Solution
    % #4 - Proof

    \noindent
    \begin{mdframed}[hidealllines=true, leftmargin=15pt, innertopmargin=0pt, innerbottommargin=0pt, skipabove=0pt, skipbelow=0pt]        
        \makebox[-5pt][r]{\large{#1)}}
        \inlinesection*{\textnormal{\emph{\normalsize{#2}}}}
    \end{mdframed}

    \addcontentsline{toc}{subsection}{#1)}

    \IfNoValueF{#3} {
        \IfBlankF{#3} {
            \sol{#3}{25}
        }
    }
    \IfNoValueF{#4} {
        \IfBlankF{#4} {
            \sol{#4}{25}
        }
    }
}

% Command for subsubquestions (e.g. i. ii. iii. etc)
\DeclareDocumentCommand\subsubquestion{ m m g g } {    
    % #1 - Question num/letter
    % #2 - Question
    % #3 - Solution
    % #4 - Proof

    \noindent
    \begin{mdframed}[hidealllines=true, leftmargin=30pt, innertopmargin=0pt, innerbottommargin=0pt, skipabove=0pt, skipbelow=0pt]  
        \makebox[-5pt][r]{\normalsize{#1)}}
        \inlinesection*{\textnormal{\emph{\small{#2}}}}
    \end{mdframed}

    \addcontentsline{toc}{subsubsection}{#1.}

    \IfNoValueF{#3} {
        \IfBlankF{#3} {
            \sol{#3}{40}
        }
    }
    \IfNoValueF{#4} {
        \IfBlankF{#4} {
            \sol{#4}{40}
        }
    }
}

% Command to make sections have connecting dots in contents
\renewcommand{\cftsecleader}{\cftdotfill{\cftdotsep}}
                   
% Command to make sections have connecting dots in contents
\renewcommand{\cftsecleader}{\cftdotfill{\cftdotsep}}
                    
\begin{document}
% \nocite{*}
\maketitle

\newpage

\tableofcontents

\newpage

% Questions and Answers
\question{1}{Functions}
\subquestion{a}{Let $X = \set{1,2,3,4}$ and $Y = \set{a,b,c,d}$. State with reason which of the rules defined below is (or is not) a function with domain $X$ and codomain $Y$.}
\subsubquestion{i}{$f(4) = a, \ f(2) = d, \ f(1) = c, \ f(3) = b$}
\sol{
    $f$ has domain of $X$ as it maps every value of $X$ to a value in $Y$. \\
    $f$ also has a codomain $Y$ as every value in $Y$ is mapped to by $f$.
}{40pt}

\subsubquestion{ii}{$g(3) = a, \ g(4) = d, \ g(1) = \ c$}
\sol{
    $g$ does not have domain of $X$ as it does not map every value of $X$ to a value in $Y$. \\
    $g$ also does not have a codomain $Y$ as every value in $Y$ is not mapped to by $g$.
}{40pt}

\subsubquestion{iii}{$h(2) = 3, \ h(3) = 1, \ h(1) = 4, \ h(4) = 2$}
\sol{
    $h$ has domain of $X$ as it maps every value of $X$ to a value in $X$. \\
    $h$ does not have a codomain of $Y$ (but rather $X$), as it maps every value in $X$, not $Y$.
}{40pt}

\newpage

\subsubquestion{iv}{$i(4) = c, \ i(1) = b, \ i(2) = a, \ i(3) = d, \ i(4) = a$}
\sol{
    $i$ has domain of $X$ as it maps every value of $X$ to a value in $Y$. \\
    $i$ does not a codomain of $Y$ as every value in $Y$ is not mapped to by $i$ (does not map to d).
}{40pt}

\newpage

\question{2}{Limits}
\subquestion{a}{Find the following limits. Explain your answer.}

\subsubquestion{i}{$\lim\limits_{n \rightarrow \infty}{\frac{2n}{n \log_{2}{\parens{\frac{1}{n}}}}}$}
\sol{
   $\lim\limits_{n \rightarrow \infty}{\frac{2n}{n \log_{2}{\parens{\frac{1}{n}}}}} = $
   $\lim\limits_{n \rightarrow \infty}{\frac{2}{\log_{2}{\parens{\frac{1}{n}}}}}$ \\
   $= \lim\limits_{n \rightarrow \infty}{\frac{2}{\log_{2}{\parens{n^{-1}}}}}$ \\
   $= \lim\limits_{n \rightarrow \infty}{-\frac{2}{\log_{2}{\parens{n}}}}$ \\
   $= -2 \lim\limits_{n \rightarrow \infty}{\frac{1}{\log_{2}{\parens{n}}}}$ \\
   $= -2\frac{\lim_{n \rightarrow \infty}{1}}{\lim_{n \rightarrow \infty}{\log_{2}{n}}}$ \\
   $= -2 \cdot \frac{1}{\infty} = -2 \cdot 0$ \\
   $= 0$
}{40pt}

\subsubquestion{ii}{$\lim\limits_{n \rightarrow \infty}{\frac{\sqrt{n^5 + 4n + 5} - \sqrt{5n^6 + 3n}}{4n^3 + 81n + 64}}$}
\sol{
    $\lim\limits_{n \rightarrow \infty}{\frac{\sqrt{n^5 + 4n + 5} - \sqrt{5n^6 + 3n}}{4n^3 + 81n + 64}} =$ \\
    $= \lim\limits_{n \rightarrow \infty}{\frac{n^3 \sqrt{\frac{1}{n} + 4\frac{1}{n^5} + 5 \frac{1}{n^6}} - n^3 \sqrt{5 + 3 \frac{1}{n^5}}}{n^3(4 + 81 \frac{1}{n^2} + 64 \frac{1}{n^3})}}$ \\
    $= \lim\limits_{n \rightarrow \infty}{\frac{\sqrt{\frac{1}{n} + 4\frac{1}{n^5} + 5 \frac{1}{n^6}} - \sqrt{5 + 3 \frac{1}{n^5}}}{4 + 81 \frac{1}{n^2} + 64 \frac{1}{n^3}}}$ \\
    $= \lim\limits_{n \rightarrow \infty}{\frac{\sqrt{0 + 4 \cdot 0 + 5 \cdot 0} - \sqrt{5 + 3 \cdot 0}}{4 + 81 \cdot 0 + 64 \cdot 0}}$ \\
    $= \lim\limits_{n \rightarrow \infty}{\frac{\sqrt{0} - \sqrt{5}}{4}}$ \\
    $= - \frac{\sqrt{5}}{4}$
}{40pt}

\newpage

\subsubquestion{iii}{$\lim\limits_{n \rightarrow \infty}{\frac{3^n + n^9}{7^n + n^2 + 5}}$}
\sol{
    $\lim\limits_{n \rightarrow \infty}{\frac{3^n + n^9}{7^n + n^2 + 5}} = $
    $\lim\limits_{n \rightarrow \infty}{\frac{\frac{3^n}{7^n} + \frac{n^9}{7^n}}{\frac{7^n}{7^n} + \frac{n^2}{7^n} + \frac{5}{7^n}}}$ \\
    $= \lim\limits_{n \rightarrow \infty}{\frac{\parens{\frac{3}{7}}^n + \frac{n^9}{7^n}}{1 + \frac{n^2}{7^n} + \frac{5}{7^n}}}$ \\
    $= \frac{\lim_{x \rightarrow \infty}{\brac{\parens{\frac{3}{7}}^n + \frac{n^9}{7^n}}}}{\lim_{x \rightarrow \infty}{\brac{1 + \frac{n^2}{7^n} + \frac{5}{7^n}}}}$ \\
    $= \frac{0 + 0}{1 + 0 + 0} = \frac{0}{1}$ \\
    $= 0$
}{40pt}

\subsubquestion{iv}{$\lim\limits_{n \rightarrow \infty}{\frac{4n + 3}{25 - 5^{\frac{4n + 3}{n^2 + 5n} + 2}}}$}
\sol{
    $\lim\limits_{n \rightarrow \infty}{\frac{4n + 3}{25 - 5^{\frac{4n + 3}{n^2 + 5n} + 2}}} =$
    $\lim\limits_{n \rightarrow \infty}{\frac{4n + 3}{25 - 5^2 \cdot 5^\frac{4n + 3}{n^2 + 5n}}}$ \\
    $= \lim\limits_{n \rightarrow \infty}{\frac{4n + 3}{25 \parens{1 - 5^\frac{4n + 3}{n^2 + 5n}}}}$ \\
    \newpage
    Note: $\lim\limits_{n \rightarrow \infty}{\frac{4n + 3}{n^2 + 5n}} = \lim\limits_{n \rightarrow \infty}{\frac{n^2 \cdot \brac{4 \cdot \frac{1}{n} + 3 \cdot \frac{1}{n^2}}}{{n^2 \cdot \brac{1 + 5 \cdot \frac{1}{n}}}}}$ \\
    $= \lim\limits_{n \rightarrow \infty}{\frac{4 \cdot \frac{1}{n} + 3 \cdot \frac{1}{n^2}}{{1 + 5 \cdot \frac{1}{n}}}}$ \\
    $= \lim\limits_{n \rightarrow \infty}{\frac{4 \cdot 0 + 3 \cdot 0}{1 + 5 \cdot 0}}$ \\
    $= \lim\limits_{n \rightarrow \infty}{\frac{0}{1}} = 0$ \\
    $\therefore \lim\limits_{n \rightarrow \infty}{\frac{4n + 3}{25 \parens{1 - 5^\frac{4n + 3}{n^2 + 5n}}}} = \lim\limits_{n \rightarrow \infty}{\frac{4n + 3}{25 \parens{1 - 5^0}}}$ \\
    $= - \infty$
}{40pt}

\newpage

\question{3}{Algorithms}
\subquestion{a}{Consider the following algorithm: \\
\widebox{        
    \textbf{Input: } A positive natural number $n$.
    \begin{enumerate}
        \item [1.] If $n \ \% \ 3 = 0$, output $n$ and stop. Otherwise, go to Step 2.
        \item [2.] If $n$ is even, replace $n$ with $n + 1$ and go back to $1.$ Otherwise, go to Step 3.
        \item [3.] Replace $n$ with $n + 2$ and go to Step 1.
        \end{enumerate}  
    } \\
    Will this algorithm run forever? Explain why or why not.
}
\sol{
    No this algorithm will not run forever. \\
    1) Firstly, the alogithm checks if the number is divisible by 3
    and if it is, it will stop. \\
    2) If n is not divisible by 3 and is even, it will add 1 to n 
    to make it odd, then return to step 1. \\
    3) If n is not divisible by 3 and is odd, it will add 2 to n 
    to make it odd again, then return to step 1. \\
    This cycle of the algorithm will always reach a number divisible by 3 
    because we are either adding 1 or 2 to n. This means that we will cover 
    all possible values for $n \mod{3}$ (those being 0, 1, 2), meaning the 
    algorithm will not run forever (as it will end when $n \mod{3} = 0)$.
}{25pt}


\subquestion{b}{Consider two algorithms, called \textbf{Algorithm A} and \textbf{Algorithm B} with the following runtimes:
\begin{gather*}
    \textbf{AlgorithmASteps}(n) = n^{10} + 4n + \log_{2}(n) \\
    \textbf{AlgorithmBSteps}(n) = 5 \log_{2}(n) + 2^{n}
\end{gather*}
}
\subsubquestion{i}{Find the run times for each algorithm for 
$n = 2$ and $n = 8$.}
\sol{
    Alogorith A: \\
    - n = 2 - $AlgorithmASteps(2) = 2^10 + 4(2) + \log_{2}(2)$ \\
    - n = 2 - $ = 1024 + 8 + 1 = 1033$ \\
    - n = 8 - $AlgorithmASteps(8) = 8^10 + 4(8) + \log_{2}(8)$ \\
    - n = 2 - $ = 1073741824 + 32 + 3 = 1073741859$ \\
    Algorithm B: \\
    - n = 2 - $AlgorithmBSteps(2) = 5 \log_{2}(2) + 2^2$ \\
    - n = 2 - $AlgorithmBSteps(2) = 5 \cdot 1 + 4 = 9$ \\
    - n = 8 - $AlgorithmBSteps(8) = 5 \log_{2}(8) + 2^8$ \\
    - n = 8 - $AlgorithmBSteps(8) = 5 \cdot 3 + 256 = 271$
}{40pt}

\subsubquestion{ii}{Which algorithm is more efficient when 
$n < 10$?}
\sol{
    When $n < 10$, Algorithm B is more efficient. You can see this from the calculations above. for both n = 2 and n = 8, Algorithm B takes significantly fewer steps than Algorithm A.
    This is because when $n < 10$, the highest power in B is always less than 10,
     while is A, the highest power is 10, meaning B will be more efficient.
}{40pt}

\subsubquestion{iii}{Which algorithm is more efficient for very large values of 
$n$? Explain your answer.}
\sol{
    For very large values of n, Algorithm A is more efficient. 
    This is because the highest power of n in Algorithm A is 10, 
    while in Algorithm B it's n. This means that as n increases,
    the efficiency of Algorithm B will decrease exponentially.
    In Big O notation, Algorithm A is $O(n^{10})$, and Algorithm B is $O(2^n)$.
}{40pt}

\end{document}